\documentclass[12pt, a4paper, titlepage]{jsarticle}

\begin{document}
\begin{titlepage}
\title{J4課題}
\author{1511179  前田 諒磨}
\maketitle
\end{titlepage}
\section{課題J4-1}
\subsection{ソースコード}
まずは、システム・コールを用いたものを載せる。
\begin{screen}
\begin{verbatim}
#include <stdio.h>
#include <stdlib.h>
#include <fcntl.h>


#define BSIZE 500
char buf[BSIZE];

void
file_copy(char *from ,char *to){
  int from_fd, to_fd;

  from_fd = open(from, O_RDONLY);

  if(from_fd < 0){
    perror( from);
    exit(1);
  }

  to_fd = open(to, O_WRONLY|O_CREAT|O_TRUNC, 0644);

 \end{verbatim}
\end{screen}
\begin{screen}
\begin{verbatim}


  if(to_fd < 0){
    perror( to );
    exit(1);
  }

  printf("from_fd =%d, to_fd = %d\n", from_fd, to_fd);
  

  while((read(from_fd, buf,BSIZE)) > 0){
    if(write(to_fd, buf, BSIZE) == -1){
      perror(to);
      exit(1);
    }
  }
}

main(int argc, char **argv)
{
  int fd, size;

  if(argc != 3){
    fprintf(stderr, "usage: %s size filename\n", argv[0]);
    exit(1);
  }

  file_copy(argv[1], argv[2]);
  close(argv[1]);
  close(argv[2]);

}

\end{verbatim}
\end{screen}


次に標準入出力ライブラリを用いたソースコードを載せる。

\begin{screen}
\begin{verbatim}
#include <stdio.h>
#include <stdlib.h>
#include <fcntl.h>

void copy(FILE *fd1, FILE *fd2){
  int c;  /*文字格納*/

  while((c = fgetc(fd1)) != EOF){
    fputc(c, fd2);
  }

}

main(int argc, char **argv){
  FILE *fd1, *fd2;

  if(argc != 3){
    fprintf(stderr,"usage: %s size filename\n", argv[0]);
    exit(1);
      }


  if((fd1 = fopen(argv[1], "r")) == NULL){
    fprintf(stderr, "%s can not open\n", argv[1]);
    exit(1);
      }
       \end{verbatim}
\end{screen}
\begin{screen}
\begin{verbatim}

  if((fd2 = fopen(argv[2], "w")) == NULL){
    fprintf(stderr, "%s can not open", argv[2]);
    exit(1);
  }

  copy(fd1, fd2);

  fclose(fd1);
  fclose(fd2);
  exit(0);
}
\end{verbatim}
\end{screen}
\subsection{実行結果}





\end{document}
